The work Johann Faulhaber and sums of powers~\cite[p. 16]{knuth1993johann} provides the following identity for
sums of odd powers
\begin{align*}
    \Sigma n^{2m-1} = \frac{1}{2m} (\bernoulli{2m}(n+1) - \bernoulli{2m})
    = \frac{1}{2m} (\coeffA{m}{0}u^{m} + \coeffA{m}{1}u^{m-1} + \cdots \coeffA{m}{m-1} u)
\end{align*}
where $\coeffA{m}{r}$ are Faulhaber's coefficients, and $u=n^2 + n$.
For every $r >m$ or $r < 0$ the coefficients $\coeffA{m}{r}$ are zeroes.
In Knuth's notation, the sigma $\Sigma n^{2m-1}$ denotes the sum of powers $\Sigma n^{2m-1} = 1^{2m-1} + 2^{2m-1} + \cdots n^{2m-1}$.
Consider the equation above with the summation limits defined explicitly
\begin{align*}
    \sum_{k=1}^{p} k^{2m-1} = \frac{1}{2m} (\coeffA{m}{0}u^{m} + \coeffA{m}{1}u^{m-1} + \cdots \coeffA{m}{m-1} u)
\end{align*}
where $u=p^2+p$.
As expected, the power sum $\sum_{k=1}^{p} k^{2m+1}$ has a closed form polynomial in $p$,
which corresponds to Faulhaber's formula.
The coefficients $\coeffA{m}{r}$ are defined by
\begin{align*}
    \coeffA{m}{k} =
    \begin{cases}
        \bernoulli{2m} & \mathrm{if} \; k=m \\
        (-1)^{m-k} \sum_{j} \binom{2m}{m-k-j} \binom{m-k+j}{j} \frac{m-k-j}{m-k+j} \bernoulli{m+k+j} & \mathrm{if} \; 0 \leq k<m \\
        0 & \mathrm{if} \; r<0 \; \mathrm{or} \; r>m
    \end{cases}
\end{align*}
For example,
\begin{table}[H]
    \begin{center}
        \setlength\extrarowheight{-6pt}
        \begin{tabular}{c|cccccccc}
            $m/r$ & 0 & 1       & 2      & 3      & 4   & 5    & 6     & 7 \\ [3px]
            \hline
            0     & 1 &         &        &        &     &      &       &       \\
            1     & 1 & 6       &        &        &     &      &       &       \\
            2     & 1 & 0       & 30     &        &     &      &       &       \\
            3     & 1 & -14     & 0      & 140    &     &      &       &       \\
            4     & 1 & -120    & 0      & 0      & 630 &      &       &       \\
            5     & 1 & -1386   & 660    & 0      & 0   & 2772 &       &       \\
            6     & 1 & -21840  & 18018  & 0      & 0   & 0    & 12012 &       \\
            7     & 1 & -450054 & 491400 & -60060 & 0   & 0    & 0     & 51480
        \end{tabular}
    \end{center}
    \caption{Coefficients $\coeffA{m}{r}$. See OEIS sequences
    ~\cite{oeis_numerators_of_the_coefficient_a_m_r,oeis_denominators_of_the_coefficient_a_m_r}.}
    \label{tab:table_of_coefficients_a}
\end{table}

In its explicit form the sum of odd powers is
\begin{align*}
    \sum_{k=1}^{p} k^{2m-1} = \frac{1}{2m} \sum_{r=0}^{m-1} \coeffA{m}{r} (p^2+p)^{m-r}
\end{align*}
Consider the examples of power sums for various values of $m$, while setting $u=p^2+p$
\begin{align*}
    \sum_{k=1}^{p} n   &= \frac{1}{2} u &&= \frac{1}{2} A_0^{(1)} u \\
    \sum_{k=1}^{p} n^3 &= \frac{1}{4} u^2 &&= \frac{1}{4} \left( A_0^{(2)} u^2 + A_1^{(2)} u \right)  \\
    \sum_{k=1}^{p} n^5 &= \frac{1}{6} \left( u^3 - \frac{1}{2} u^2 \right) &&= \frac{1}{6} \left( A_0^{(3)} u^3 + A_1^{(3)} u^2 + A_2^{(3)} u \right) \\
    \sum_{k=1}^{p} n^7 &= \frac{1}{8} \left( u^4 - \frac{4}{3} u^3 + \frac{2}{3} u^2 \right) &&= \frac{1}{8} \left( A_0^{(4)} u^4 + A_1^{(4)} u^3 + A_2^{(4)} u^2 + A_3^{(4)} u \right)
\end{align*}
\begin{align*}
    \sum_{k=1}^{p} n
    &= \frac{1}{2} \cdot 1 \cdot (p^2 + p) = \frac{1}{2} \left(p^2+p\right) \\[8pt]
    \sum_{k=1}^{p} n^3
    &= \frac{1}{4} \left( 1 \cdot (p^2 + p)^2 + 0 \cdot (p^2 + p) \right) = \frac{1}{4} \left(p^2+p\right)^2 \\[8pt]
    \sum_{k=1}^{p} n^5
    &= \frac{1}{6} \left( 1 \cdot (p^2 + p)^3 - \frac{1}{2} \cdot (p^2 + p)^2 + 0 \cdot (p^2 + p) \right) = \frac{1}{6} \left(\left(p^2+p\right)^3-\frac{1}{2} \left(p^2+p\right)^2\right) \\[8pt]
    \sum_{k=1}^{p} n^7
    &= \frac{1}{8} \left( 1 \cdot (p^2 + p)^4 - \frac{4}{3} \cdot (p^2 + p)^3 + \frac{2}{3} \cdot (p^2 + p)^2 + 0 \cdot (p^2 + p) \right) \\
    &=\frac{1}{8} \left(\left(p^2+p\right)^4-\frac{4}{3} \left(p^2+p\right)^3+\frac{2}{3} \left(p^2+p\right)^2\right)
\end{align*}

Mathematica functions to validate, see this \href{https://github.com/kolosovpetro/faulhabers-coefficients-examples}{\texttt{GitHub repository}}
\begin{itemize}
    \item \texttt{FaulhaberCoefficients[n,k]} validates the coefficients $\coeffA{m}{r}$
    \item \texttt{FaulhaberSum[p,m]} validates the identity $\sum_{k=1}^{p} k^{2m-1} = \frac{1}{2m} \sum_{r=0}^{m-1} \coeffA{m}{r} (p^2+p)^{m-r}$
    \item \texttt{SumOfOddPowers[p, m]} power sum $\sum_{k=1}^{p} k^{2m-1}$, the result matches with $\sum_{k=1}^{p} k^{2m-1} = \frac{1}{2m} \sum_{r=0}^{m-1} \coeffA{m}{r} (p^2+p)^{m-r}$
\end{itemize}
